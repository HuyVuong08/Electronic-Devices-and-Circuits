\documentclass[12pt]{extarticle}
%Some packages I commonly use.
\usepackage[english]{babel}
\usepackage{graphicx}
\usepackage{framed}
\usepackage[normalem]{ulem}
\usepackage{amsmath}
\usepackage{amsthm}
\usepackage{amssymb}
\usepackage{amsfonts}
\usepackage{enumerate}
\usepackage[utf8]{inputenc}
\usepackage[top=1 in,bottom=1in, left=1 in, right=1 in]{geometry}

%A bunch of definitions that make my life easier
\newcommand{\matlab}{{\sc Matlab} }
\newcommand{\cvec}[1]{{\mathbf #1}}
\newcommand{\rvec}[1]{\vec{\mathbf #1}}
\newcommand{\ihat}{\hat{\textbf{\i}}}
\newcommand{\jhat}{\hat{\textbf{\j}}}
\newcommand{\khat}{\hat{\textbf{k}}}
\newcommand{\minor}{{\rm minor}}
\newcommand{\trace}{{\rm trace}}
\newcommand{\spn}{{\rm Span}}
\newcommand{\rem}{{\rm rem}}
\newcommand{\ran}{{\rm range}}
\newcommand{\range}{{\rm range}}
\newcommand{\mdiv}{{\rm div}}
\newcommand{\proj}{{\rm proj}}
\newcommand{\R}{\mathbb{R}}
\newcommand{\N}{\mathbb{N}}
\newcommand{\Q}{\mathbb{Q}}
\newcommand{\Z}{\mathbb{Z}}
\newcommand{\<}{\langle}
\renewcommand{\>}{\rangle}
\renewcommand{\emptyset}{\varnothing}
\newcommand{\attn}[1]{\textbf{#1}}
\theoremstyle{definition}
\newtheorem{theorem}{Theorem}
\newtheorem{corollary}{Corollary}
\newtheorem*{definition}{Definition}
\newtheorem*{example}{Example}
\newtheorem*{note}{Note}
\newtheorem{exercise}{Exercise}
\newcommand{\bproof}{\bigskip {\bf Proof. }}
\newcommand{\eproof}{\hfill\qedsymbol}
\newcommand{\Disp}{\displaystyle}
\newcommand{\qe}{\hfill\(\bigtriangledown\)}
\setlength{\columnseprule}{1 pt}


\title{Math 335 Portfolio}
\author{Jean Marie Linhart}
\date{January 2019}

\begin{document}

\maketitle

\section{Induction Proofs}
\subsection{Ordinary Induction}
\begin{normalsize}
According to Kirchoff's Voltage Laws (KVL), we have the following equation:
\begin{align*}
&-12 + V_{R_3} + 3V_F = 0\\
&\Longrightarrow V_{R_3} = 12 - 3V_F\\
&\Longrightarrow R_3 = \frac{V_{R_3}}{I_{sccR_3}} = \frac{V_{R_3}}{I_F} = \frac{12 - 3V_F}{I_F}
\end{align*}
$\bullet$ In case of $V_F  = 3V \longrightarrow 3.4V$
\begin{align*}
&I_F = 25mA - 30mA\\
&\Longrightarrow R_3 \in \left[\frac{12 - 3x3.4}{30x10^{-3}} ; \frac{12 - 3x3}{25x10^{-3}}\right]\\
&\iff R_3 \in [60 ; 120] (\Omega)
\end{align*}
$\bullet$ Choose $R_1 = 40000\Omega$, we have the following expression:
\begin{align*}
&I_{R_2} = I_L = \frac{0.7}{40000} = 1.78x10^6(A)
\end{align*}
Results below can be extracted by appling kirchoff's Voltage Laws. 
\begin{align*}
&-12 + V_{R_2} + V_{R_L} = 0\\
&\Longrightarrow V_{R_2} = 12 - V_{R_L} = 12 - 0.7 = 11.3V\\
&\Longrightarrow R_2 = \frac{V_{R_2}}{I_{R_2}} = 645414 (\Omega)
\end{align*}
Base on the value of $R_L$ measured by VOM, the following results can be infered:
\begin{align*}
&R_2 = \frac{11.3}{\frac{0.7}{R_L}} = \frac{11.3 X R_L}{0.7} = 12.14R_L (\Omega)
\end{align*}
Applying Kirchoff's Laws:
\begin{align*}
&I_{R_1} = I_{R_2} + I_{R_3}\\
&\Longrightarrow I_{R_1} = I_{R_2} + I_F\\
&\Longrightarrow I_{R_1} = \frac{0.7}{R_2} + I_F\\
&\Longrightarrow I_{R_1} \in [0.025 ; 0.03] (A)
\end{align*}
Measurement results in laboratory reportedly show that voltage at the two ends of the capacitor varies around $12\sqrt{2} (V)$ 
\begin{align*}
V_{0C} = 12\sqrt{2} (V)
\end{align*}
$\bullet$ In case of the worst situation when $V_{DC} = 18.8(V)$
$R_1$ is designed in ways such that $V_{DC} = 12 (V)$
\begin{align*}
&\Longrightarrow R_1 = \frac{18.8 - 12}{I_{R_1}}\\
&\Longrightarrow R_1 \in [227 ; 275]
\end{align*}
Power: Power Index $\geq 1.5$
In case of $V_{DC}$ exceeds the common voltage of $12V$, the circuit can withstand up to $18V$ before suffering structural damages.
\begin{align*}
&\Longrightarrow V_{DC} without R_1 = 18\sqrt{2}\\
&\Longrightarrow P_{R_1} = \frac{(18\sqrt{2} - R_1)^2}{R_1}\\
&\Longrightarrow R_1 \in [227 ; 211]\\
\end{align*}
Procedure:\\
1. List of Components\\
- Zener Diode\\
- 100$\Omega$ Resistor\\
- 470000$\Omega$ Resistor\\
- 3 LEDs\\
- C1815 NPN Transistor\\
- Light Sensoring Resistor\\
- 4 Diodes\\
2. $R_1$, $R_2$, $R_3$ Build Method\\
%Ve 2 mach R1 R2 dum t nha m
$R_3 = R = 100(\Omega)$.\\
\end{normalsize}
\end{document}